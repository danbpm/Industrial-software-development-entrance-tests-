%% LyX 2.4.2.1 created this file.  For more info, see https://www.lyx.org/.
%% Do not edit unless you really know what you are doing.
\documentclass[russian]{article}
\usepackage[T2A]{fontenc}
\usepackage[utf8]{luainputenc}
\usepackage{cprotect}
\usepackage{float}
\usepackage{amsmath}

\makeatletter

%%%%%%%%%%%%%%%%%%%%%%%%%%%%%% LyX specific LaTeX commands.
\DeclareRobustCommand{\cyrtext}{%
  \fontencoding{T2A}\selectfont\def\encodingdefault{T2A}}
\DeclareRobustCommand{\textcyrillic}[1]{\leavevmode{\cyrtext #1}}

\floatstyle{ruled}
\newfloat{algorithm}{tbp}{loa}
\providecommand{\algorithmname}{Алгоритм}
\floatname{algorithm}{\protect\algorithmname}

\makeatother

\usepackage{babel}
\usepackage{listings}
\renewcommand{\lstlistingname}{Листинг}

\begin{document}
\title{Задача 5}
\maketitle

\subsection*{Условие: }

Дан треугольник (в виде списка списков, где $triangle[i]$ --- это
строка с $i+1$ элементом). Найди минимальную сумму пути от вершины
треугольника до его основания. На каждом шаге ты можешь переместиться
на соседнее число в строке ниже. Если ты находишься на индексе $i$
в текущей строке, ты можешь перейти на индекс $i$ или индекс $i+1$
в следующей строке. Предложи решение с алгоритмической сложностью,
не превышающей $O(n^{2})$.

\subsection*{Решение: }

Условие можно переформулировать в постановке задачи динамического
программирования.

\subsubsection*{Формализация состояний: }

Пусть состояние $(i,j)$ обозначает ячейку в строке $i$ и столбце
$j$, где $i=0,1,2,n-1;$ $j=0,1,...,i$.

\subsubsection*{Функция Беллмана: }

Пусть $S(i,j)$ -- минимальная сумма пути от ячейки $(i,j)$ до любой
ячейки основания. Тогда уравнение Беллмана имеет вид:

\[
S(i,j)=\begin{cases}
triangle[i][j] & ,\text{ если }i=n-1\text{(основание})\\
triangle[i][j]+\min\{S(i+1,j),S(i+1,j+1)\} & ,\text{ иначе}
\end{cases}.
\]
Минимальному расстоянию от вершины треугольника до основания будет
соответствовать значение $S(0,0)$.

\subsubsection*{Алгоритм:}

Алгоритм начинает проход по всем ячейкам треугольника для поиска минимального
пути от каждой вершины до основания, начиная с предпоследней строки.
Результаты записываются в исходный список, тем самым изменяя значение
входного параметра функции, переданного по ссылке. По завершению работы
внешнего цикла возвращается значение самого первого элемента списка,
который представляет собой $S(0,0)$, что соответствует описанию алгоритма,
приведенного выше.

\begin{algorithm}
\bgroup
\begin{lstlisting}
def minimum_total(triangle: list[list[int]])->int: 
    rowsCount = len(triangle)
    # идём снизу вверх по строкам, начиная с предпоследней строки 
    for i in range(rowsCount - 2, -1, -1): 
        for j in range(len(triangle[i])): 
            # вычисляем значение функции Беллмана S(i,j)
            triangle[i][j] += min(triangle[i+1][j], 
                                  triangle[i+1][j+1])
    return triangle[0][0]
\end{lstlisting}
\leavevmode\egroup

\caption{Листинг функции на языке Python 3}
\end{algorithm}


\subsubsection*{Сложность:}

Приведем оценку асимптотической сложности алгоритма в зависимости
от размера входного списка $n$. Оценка асимптотической сложности
фактически сводится к расчету количества всех итераций двойного цикла,
поскольку время работы функции внутри цикла для всех итераций можно
оценить некоторой положительной константой, а также добавление константных
членов вне цикла не повлияет на асимптотическое поведение. Запишем
количество итераций цикла:
\[
\sum_{i=0}^{n-1}\sum_{j=0}^{i}1=\sum_{i=0}^{n-1}(i+1)=\frac{n\cdot(n+1)}{2}=\Theta(n^{2}),\text{ при }n\rightarrow\infty.
\]
Точная асимптотическая оценка данного алгоритма равна $\Theta(n^{2}).$ 
\end{document}
